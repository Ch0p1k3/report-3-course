\section{Результаты и дальнейшие планы}
\label{section.output}
В результате выполнения данной работы были реализованы smart-контракты NFT (по стандарту NEP-171) и маркетплейса, реализован discord-бот, который предоставляет удобный интерфейс пользователю и выполняет роль NFT маркетплейса с функционалом, который планировался изначально, а также реализован сервис для генерации NFT, доступ к которому, пользователь получает через discord-бот.

В качестве дальнейшей работы планируется пересмотр текущего распределенного хранилища, в котором хранятся метаданные и медиа-файл NFT; исправление выявленных багов в discord-боте; добавление нового функционала:
\begin{enumerate}
    \item Создание и привязка токена к коллекциям аналогично paras. В процессе разработки было обнаружено, что такая структура является наиболее удобной, поэтому в будущем данная идея будет позаимствована у paras.
    \item Агрегация токенов по убыванию/возрастанию цены, времени создания, редкости (в случае токенов генерируемыми генеративными моделями).
    \item Возможность сделать оффер (предложение о покупке) на любой токен(как продаваемый, так и не продаваемый). Это очень удобно, например, когда пользователю цена кажется слишком большой и он решает предложить цену меньше, или когда токен не продается и пользователь большой ценой хочет его выкупить.
    \item Сбор статистики о количестве созданных/уничтоженных токенов, объеме продаж за день/месяц/год, количестве сделок за день/месяц/год. Вышеописанную статистику можно собирать также и по конкретному пользователю.
    \item Добавление большего числа генеративных моделей.
\end{enumerate}

\newpage
