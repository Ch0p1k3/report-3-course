% \anonsection{Ожидаемые результаты}
% Ожидаемым результатом является: NFT маркетплейс в Discord посредствам взаимодействия с ботом, которые имеет следующий функционал:
% \begin{itemize}
%     \item Реализация меню как и по средствам команд в чате, так и интерактивно с помощью функционала представленным Discord API(slash commands, buttons, select menus);
%     \item Взаимодействия с ботом:
%     \begin{itemize}
%         \item Авторизация в NEAR wallet;
%         \item Информация о кошельке: текущее количество NEAR, список NFT;
%         \item Просмотри списка продаваемых NFT. Агрегация, фильтрация данного списка. Покупка продаваемой NFT;
%         \item Просмотр списка самых дорогих NFT проданных на площадке;
%         \item Продажа имеющихся NFT: моментальная продажа;
%         \item Реализация системы(модели), которая предлагает цену NFT;
%         \item GAN, который создает NFT;
%         \item Система конкурсов: возможность разыграть NFT в области канала Discord;
%     \end{itemize}
%     \item Вся логика smart-контрактов NFT должна удовлетворять стандарту;
%     \item Максимально большое покрытие тестами всего кода;
% \end{itemize}

% \anonsection{План работ}
% \begin{tabular}{|p{0.2cm}|p{7.5cm}|p{7.5cm}|}
%     \hline
%     №& Дата& Содержание этапа работы\\\hline
%     1& 10.01.2022& Изучение NEAR Protocol и языка Rust для написания smart-контрактов, практика в написании некоторых простых примеров smart-контрактов, изучение NodeJS TypeScript для <<frontend>>\\\hline
%     2& 20.01.2022& Изучение стандарта NFT в Near Protocol\\\hline
%     3& 01.02.2022& Написание скелета бота на базе Discord API, используя NodeJS/Typescript.\\\hline
%     4& 10.02.2022& Добавление модуля авторизации через NEAR кошелек.\\\hline
%     5& 25.03.2022& Реализация NFT smart-контракта вместе с тестами.\\\hline
%     6& 05.04.2022&Добавление функций покупки/продажи NFT в Discord боте.\\\hline
%     7& 15.04.2022& Сбор признаков для рекомендательной системы\\\hline
%     8& 30.04.2022& Обучение и внедрение моделей рекомендательной системы или GAN в
%     Discord бота.\\\hline
%     9& -& Предъявление итоговой версии командного проекта.\\\hline

% \end{tabular}

\newpage
