\section*{ХОД РАБОТЫ}

\begin{enumerate}
\item Поместила в рабочую область Cisco Packet Tracer одиночный
персональный компьютер. 
  \begin{figure}[htbp]
    \centering
    \includegraphics[width=125mm,height=100mm]{fig/step1.png}
    \caption*{\\[-35 pt] \it Одиночный персональный компьютер}
  \end{figure}
Открыла вкладку Physical окна настроек данного
устройства и последовательно добавила в него все доступные аппаратные
модули. 
\begin{figure}[h!]
\begin{multicols}{2}
\hfill
\includegraphics[width=70mm]{fig/step2.png}
\hfill
\caption*{\\[-35 pt] \it Доступные аппаратные модули}
\label{figLeft}
\hfill
\includegraphics[width=70mm]{fig/step3.png}
\hfill
\caption*{\\[-35 pt] \it Компьютер с подключенными наушниками и микрофоном}
\label{figRight}
\end{multicols}
\end{figure}
  
 (Краткое описание аппаратных модулей см. в Приложении)

\item Поместила в рабочую область Cisco Packet Tracer второй
персональный компьютер. Соединила их с использованием кабеля,
предназначенного для соединения пары устройств данного типа. 
  \begin{figure}[htbp]
    \centering
    \includegraphics[width=120mm,height=90mm]{fig/step4.png}
    \caption*{\\[-35 pt] \it Два соединенных персональных компьютера}
  \end{figure}
  
Открыла форму настроек интерфейса FastEthernet0 для обоих компьютеров. Исследовала
первые три позиции в списке настроек:
	\begin{itemize}
	\item Port Status --- состояние порта (включен/выключен)
	\item Bandwidth --- пропускная способность канала (100/10 Мбит)
	\item Duplex --- режим работы приемо-передающих устройств (на прием и передачу/либо на прием, либо на передачу)
	\end{itemize}
Определила, что при одних сочетаниях настроек соединение будет установлено, а при других --- нарушено.
\begin{figure}[h!]
\begin{multicols}{2}
\hfill
\includegraphics[width=70mm]{fig/step5.png}
\hfill
\caption*{\\[-35 pt] \it <<Рабочие>> сочетания настроек}
\label{figLeft}
\hfill
\includegraphics[width=70mm]{fig/step6.png}
\hfill
\caption*{\\[-35 pt] \it Настройки, не позволяющие установить соединение}
\label{figRight}
\end{multicols}
\end{figure}

Соединение устанавливается только тогда, когда позиции 2 и 3 настроены одинаково, а Port Status = <<On>>.

\item Создала сеть из двух компьютеров с установленным соединением.
Назначила им IP-адреса 192.168.0.1 и 192.168.0.2. Проверила
соединение с помощью утилиты ping с обоих компьютеров.\\
  \begin{figure}[h!]
    \centering
    \includegraphics[width=150mm,height=80mm]{fig/step7.png}
    \caption*{\\[-35 pt] \it Соединение успешно установлено}
  \end{figure}

Поменяла IP-адрес одного из компьютеров и в каждом случае проверила соединение с помощью утилиты ping:
	\begin{enumerate}[a)]
	\item 192.168.0.3 --- соединение установлено;
	\item 192.168.1.3 --- соединение нарушено, так как адреса
	находятся в разных подсетях (маска 255.255.255.0 делает адресом подсети 3 первых октета, а в случае б) они не равны)
	\end{enumerate}
	
\begin{figure}[h!]
\begin{multicols}{2}
\hfill
\includegraphics[width=70mm]{fig/step8.png}
\hfill
\caption*{\\[-40 pt] \it С IP-адресом 192.168.0.3 всё хорошо}
\label{figLeft}
\hfill
\includegraphics[width=70mm]{fig/step8b.png}
\hfill
\caption*{\\[-40 pt] \it А вот с 192.168.1.3 всё сломалось}
\label{figRight}
\end{multicols}
\end{figure}
\item Поместила в рабочую область Cisco Packet Tracer два компьютера и
концентратор типа Hub-PT.

При соединении компьютеров с помощью прямого кабеля через концентратор соединение было успешно установлено. \\
  \begin{figure}[h!]
    \centering
    \includegraphics[width=130mm]{fig/step9.png}
    \caption*{\\[-40 pt] \it Устройства, соединенные через концентратор}
  \end{figure}

\item Создала сеть из двух компьютеров, соединенных через
концентратор. Назначила устройствам PC0 и PC1 IP-адреса 192.168.0.1 и 192.168.0.2 соответственно и проверила соединение с
помощью утилиты ping с обоих компьютеров.\\
  \begin{figure}[h!]
    \centering
    \includegraphics[width=150mm]{fig/step10.png}
    \caption*{\\[-40 pt] \it Соединение успешно установлено!}
  \end{figure}

\end{enumerate}