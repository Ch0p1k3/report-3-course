\section{Введение}
\subsection{Актуальность и значимость}

\subsection{Постановка задачи}
В качестве блокчейна используется NEAR protocol\cite{nearprotocol_2022}. NEAR Protocol работает по схеме Proof-of-Stake(Pos). Отличительные черты относительно других блокчейнов - улучшенная масштабируемость, производительность, а также простота реализации приложений.

\begin{definition}
    Блокчейн - децентрализованная база данных, которая содержит информацию о всех операциях произведенных в ней.
    Информация об операциях хранится в виде цепочки блоков.  Удалить или изменить цепочку блоков невозможно, все это защищено криптографическими методами. Самым первым блокчейном является Bitcoin~\cite{nakamoto2012bitcoin}.
\end{definition}

\begin{definition}
    DApps --- это приложения, которые включают логику работы с функциями блокчейна~\cite{ramamurthy2020blockchain}.
\end{definition}

Самой значимой частью реализации DApp являются Smart-контракты. Копии Smart-контрактов разворачивается с помощью специальной транзакции на всех узлах-участниках и исполняются в сети блокчейна.

\begin{definition}
    Smart-контракт --- это неизменяемый исполняемый код, представляющий логику DApp, работающий в блокчейне~\cite{ramamurthy2020blockchain}. Часто сокращают до слова контракт. В некоторых протоколах называют по-другому, например в Solana - это программы\cite{solanaprogramlibrarydocs}.
\end{definition}

\begin{definition}
    Транзакция — это наименьшая единица работы, которая может быть назначена сети блокчейна. Работа в данном случае означает вычисление(выполнение функции) или хранение(чтение/запись данных)\cite{neardocumentationtransaction}.
\end{definition}

\begin{definition}
    Узлы-участники/валидаторы - множество машин, которое обрабатывает транзакции в блокчейне.
\end{definition}

Для написания smart-контрактов Near protocol предоставляет sdk на языках Rust и AssemblyScript(near-sdk-rs\cite{nearsdkrs} и near-sdk-as\cite{nearsdkas} соответственно). В данном проекте smart-контракты NFT и маркетплейса реализовываются на языке Rust.

Каждый smart-контракт в Near(написанный на Rust/Assembly Script) переводится в WebAssembly(Wasm), который непосредственно исполняет виртуальная машина на участвующем узле(валидаторе) блокчейна. У smart-контракта, есть два вида функций: которые меняют состояние блокчейна - <<change operations>> и так называемые <<view operations>>, которые не меняют состояние машины, из названия данных операций можно понять, что первый вид операций, что-то сохраняет в блокчейн, а другая получает некоторую информацию с блокчейна, то есть readonly операция. Каждая операция имеет некоторую стоимость, которая измеряется в <<Gas>> ~\cite*{ramamurthy2020blockchain, docsnear}. Также есть <<payable>> операции, которые запрашивают некоторую сумму токена, но это больше не как вид функций, а дополнение к ним.

\begin{remark}
    Gas: сборы на исполнение транзакции не рассчитываются в токенах NEAR, вместо это она рассчитываются через Gas. Преимущество в том, что данные единицы - детерминированы, то есть одна и та же транзакция будет всегда будет стоить одинаковое количество Gas. Стоимость Gas пересчитывается в зависимости от загруженности сети в блокчейне~\cite*{docsnear}.
\end{remark}

Для того, чтобы уметь работать с NFT, нужно написать соответствующий Smart-контракт, он базируется на описанном стандарте в спецификации Near Protocol~\cite*{docsnear, nearspec}.

% \begin{definition}
%     GAN(Generative adversarial network) - алгоритм машинного обучения без учителя, которая позволяет генерировать фотографии. Позднее были изучение и иные генеративные модели, которые умеют генерировать не только фотографии, но и например музыку.
% \end{definition}

\subsection{Этапы проекта}
В рамках групповой курсовой работы была поставлена цель реализации discord-бота с функционалом NFT маркетплейса в NEAR protocol и сервисом генерации NFT, используя генеративно-состязательную сеть. Для реализации данной цели были выделены следующие этапы:
\begin{itemize}
    % \item Ознакомиться и понять NEAR Protocol, в частности выучить язык Rust для написания
    % smart-контрактов. Реализовать некоторые несложные примеры smart-контрактов. Выучить nodejs/typescript для того, чтобы реализовать взаимодействия пользователя со smart-контрактами.
    % \item Разобраться с Discord API, в частности с библиотеками discord.js/discord.ts;
    % \item Написать код smart-контракта на языке Rust, который будет описывать логику NFT;
    % \item Написать код Discord бота, который будет предоставлять удобный интерфейс взаимодействия пользователю;
    % \item На основе признаков обучить и внедрить модель, которая будет рекомендовать пользователю купить NFT. Данной задачей по большей мере будет заниматься другой участник группы;
    % \item По возможности обучить gan, чтобы пользователь мог создавать NFT. Аналогично, по большей мере должен реализовать иной человек с команды;
    % \item Проанализировать и представить полученные результаты.
    \item Изучить теоретический базис связанный с NEAR Protocol(Лущ, Басалаев, Токкожин, Кусиденов)
    \item Реализовать smart-контракты(Басалаев):
    \begin{itemize}
        \item
    \end{itemize}
    \item Разработать discord-бота(Лущ):
    \begin{itemize}
        \item Изучить Javascript/Typescript;
        \item Изучить основы работы с браузером через Javascript(сессионное/локальное хранилище браузера, класс window);
        \item Изучить near-api-js и его кода для дальнейшего его переписывания под функциональность discord;
        \item Реализовать KeyStore\cite{nearclasskeystore} работающий через Redis\cite{redis};
        \item Написать реализацию авторизации в Near Wallet\cite{nearwallet} через discord-бота, который использует вышеописанный KeyStore;
        \item Написать реализацию создания url на подпись транзакции/транзакций(одна транзакция\footnote{В данном контексте класс Transaction\cite{nearclasstransaction}}, один Action\cite{nearclassaction}; одна транзакция, несколько Action; несколько транзакций, несколько Action);
        \item Изучение децентрализованных распределенных хранилищ;
        \item Создание <<Профиля пользователя>>(Вызов осуществляется через slash-команду\cite{discordjsslashcommands} или контекстное меню\cite{discordtscontextmenu});
        \item Реализация просмотра списка NFT, которыми владеет пользователь, которые продает пользователь, которые продаются на всем маркетплейсе(Вызовы осуществляются через контекстные меню, slash-команды, кнопки в профиле пользователя. Список выглядит по-разному в зависимости от количества NFT, если NFT много, то будет подгружаться только часть в целях оптимизации ресурсов);
        \item Поддержка покупки, продажи, отмены продажи NFT(Вызовы в виде кнопок при просмотре NFT списка);
        \item Поддержка изменения цены NFT // пока что не сделано, но сделан метод в smart-контракте;
        \item Поддержать сервис GAN в discord-bot // пока что не сделано;
        \item Сделать docker образ для удобного деплоя discord-бота;
        \item Деплой discord-бота на облачный сервис(Кусиденов);
    \end{itemize}
    \item Реализовать GAN(Токкожин):
    \item Реализовать сервис GAN(Кусиденов):
\end{itemize}

\subsection{Структура работы}

