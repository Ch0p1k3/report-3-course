\newglossaryentry{discretization}{name=Дискретизация, description={представление непрерывной функции выборкой её значений; здесь --- переход от функции на римановом многообразии к функции на графе.}}
\newglossaryentry{d}{name=d, description={dd}}

\anonsection{Постановка задачи}
    Для того, чтобы заняться платформой требовалось выбрать блокчейн, посредствам которого все это будет реализовано, мы выбрали NEAR protocol\footnote{\url{https://near.org/}}. NEAR protocol - это децентрализованная платформа, которая обеспечивает идеальную среду для разработки DApps.
    
    \begin{definition}
        DApps --- это приложения, которые включают логику работы с функциями блокчейна~\cite{ramamurthy2020blockchain}.
    \end{definition}

    NEAR protocol работает по схеме Proof-of-Stake(Pos), от других блокчейнов, его отличает большая пропускная способность, скорость, улучшенная масштабируемость а также, что для нас стало решающим фактором - это его дружественность к разработчикам(developer friendly) и предоставляет огромное количество источников для изучения их инструментария.

    В DApps самой значимой частью кода являются Smart-контракты. Копии Smart-контрактов разворачивается с помощью специальной транзакции на всех узлах-участниках(валидаторов) и исполняются в сети блокчейна.

    \begin{definition}
        Smart-контракт --- это неизменяемый исполняемый код, представляющий логику DApp, работающий непосредственно в блокчейне~\cite{ramamurthy2020blockchain}. Часто сокращают до слова контракт. В некоторых протоколах называют по-другому, например в Solana\footnote{\url{https://solana.com/}} - это программы\footnote{\url{https://spl.solana.com/}}.
    \end{definition}

    В разных блокчейнах - разный язык программирования для написания Smart-контрактов. Near protocol предоставляет некоторый функционал для написания Smart-контрактов на языках Rust и AssemblyScript~\cite{docsnear}(near-sdk-rs\footnote{\url{https://github.com/near/near-sdk-rs}} и near-sdk-as\footnote{\url{https://github.com/near/near-sdk-as}} соответственно). Авторы не рекомендуют использовать AssemblyScript, отдавая свое предпочтение больше Rust для написания контрактов.

    Каждый smart-контракт в Near(написанный на Rust/Assembly Script) переводится в WebAssembly(Wasm), который непосредственно исполняет виртуальная машина на участвующем узле(валидаторе) блокчейна. У smart-контракта, есть два вида функций: которые меняют состояние блокчейна - <<change operations>> и так называемые <<view operations>>, которые не меняют состояние машины, из названия данных операций можно понять, что первый вид операций, что-то сохраняет в блокчейн, а другая получает некоторую информацию с блокчейна, то есть readonly операция. Каждая операция имеет некоторую стоимость, которая измеряется в <<Gas>> ~\cite*{ramamurthy2020blockchain, docsnear}. Также есть <<payable>> операции, которые запрашивают некоторую сумму токена, но это больше не как вид функций, а дополнение к ним.

    \begin{remark}
        Gas: сборы на исполнение транзакции не рассчитываются в токенах NEAR, вместо это она рассчитываются через Gas. Преимущество в том, что данные единицы - детерминированы, то есть одна и та же транзакция будет всегда будет стоить одинаковое количество Gas. Стоимость Gas пересчитывается в зависимости от загруженности сети в блокчейне~\cite*{docsnear}.
    \end{remark}

    Для того, чтобы уметь работать с NFT, нужно написать соответствующий Smart-контракт, он базируется на описанном стандарте в спецификации Near Protocol~\cite*{docsnear, nearspec}.

    Чтобы реализовать данный проект были поставлены следующие задачи:
    \begin{itemize}
        \item Ознакомиться и понять NEAR Protocol, в частности выучить язык Rust для написания
        smart-контрактов. Реализовать некоторые несложные примеры smart-контрактов. Выучить nodejs/typescript для того, чтобы реализовать взаимодействия пользователя со smart-контрактами.
        \item Разобраться с Discord API, в частности с библиотеками discord.js/discord.ts;
        \item Написать код smart-контракта на языке Rust, который будет описывать логику NFT;
        \item Написать код Discord бота, который будет предоставлять удобный интерфейс взаимодействия пользователю;
        \item На основе признаков обучить и внедрить модель, которая будет рекомендовать пользователю купить NFT. Данной задачей по большей мере будет заниматься другой участник группы;
        \item По возможности обучить gan, чтобы пользователь мог создавать NFT. Аналогично, по большей мере должен реализовать иной человек с команды;
        \item Проанализировать и представить полученные результаты.
    \end{itemize}
    \begin{definition}
        GAN(Generative adversarial network) - алгоритм машинного обучения без учителя, которая позволяет генерировать фотографии. Позднее были изучение и иные генеративные модели, которые умеют генерировать не только фотографии, но и например музыку.
    \end{definition}