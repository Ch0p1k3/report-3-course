\anonsection{Обзор существующих работ и решений}

    На сегодняшний день NFT маркетплейсов огромное количество, на блокчейне Ethereum, как примеры: opensea\footnote{https://opensea.io/}, rarible\footnote{https://rarible.com/}, на Solana: solanart\footnote{https://solanart.io/}. На Near protocol, на данный момент самыми популярными являются: Paras\footnote{https://paras.id/}, Mintbase\footnote{https://www.mintbase.io/}.

    Из двух представленных маркетплейсов, наиболее популярным является Paras. Он представляет базовый функционал маркетплейса: аутентификация в кошелек; покупка, продажа, просмотр NFT. У Paras также есть свой токен - Paras. В Paras представлены следующие медиа файлы: картины, фотографии, пиксель-арты.

    Хоть и Mintbase менее популярен, чем Paras, но представляет гораздо больше видов медиа объектов: 3D модели, gif-изображения, аудио объекты. Но сама подача(оформление сайта) выглядит гораздо хуже, чем в Paras.

    Объединяет данные маркетплейсы то, что у них единый стандарт реализации NFT, предоставленный Near protocol~\cite*{nearspec}. Уже как 3 года есть стандарт ERC-721\footnote{https://ru.bitcoinwiki.org/wiki/ERC-721}, который был утвержден сообществом Ethereum. Стандарт в Near немного расширяет возможности, из-за уникальных возможностей реализованных в Near protocol~\cite*{nearspec}. Наша команда планирует придерживаться данного стандарта и реализовывать его, возможно потребуется некоторое его расширение, но об этом пока неизвестно. Важным вопросом является хранение в блокчейне самого NFT.

    В данном стандарте количество, балансы ограничены 128-битным беззнаковым целочисленным типом. Вся сериализация, десериализация происходит с помощью JSON. Сам контракт отслеживает изменения в хранилище.

    NFT хранит следующую информацию о себе: сам идентификатор токена, ID владельца и метаданные этого токена. Метаданные делятся на типа: первый класс, который хранит версию, название токена, ссылка на информацию, которая представляется в виде JSON, ссылка на sha256 хеш. Другой класс хранит: название, описание, время создания, время истечение представленное в UNIX epoch.

    В Smart-контракте NFT должны быть реализованы следующие функции:

    \begin{itemize}
        \item nft\_transfer - <<change operation> функция передачи NFT от одного аккаунта другому.;
        \item nft\_transfer\_call - такая же <<change operation>> функция, что и предыдущая, за исключением того, что вызывается nft\_on\_transfer при удачной передачи NFT и nft\_resolve\_transfer при неудачной попытке;
        \item nft\_token - <<view operation>> функция, которая по ID токена возвращает всю информацию о нем.
    \end{itemize}
