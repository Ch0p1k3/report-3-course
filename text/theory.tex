\section{Обзор существующих работ и решений}
\label{section.3}
\subsection{Маркетплейсы}
НВ данный момент существует большое количество NFT маркетплейсов: opensea\cite{opensea}, rarible\cite{rarible}, solanart\cite{solanart}. Если брать маркетплейсы только на базе NEAR Protocol, тогда существуют такие примеры как: Paras\cite{paras}, Mintbase\cite{mintbase}, остановимся на них поподробнее.

\paragraph{Paras}

Paras является наиболее популярным и представляет следующий набор функций:
посмотреть какие NFT выставлены на продажу, посмотреть купленные NFT, купить NFT, продать NFT, создать новую коллекцию NFT.
На площадке представлены следующие виды NFT: пиксель-арты, иллюстрации, абстрактные картины, картины разных персонажей, фотографии. Все объекты можно отсортировать по убыванию или возрастанию цены.

Smart-контракты Paras лежат в открытом доступе\cite{parasnftcontract, parasmarketplacecontract}.

// Тут что-то про smart-контракты

Paras использует сервис fleek, этот 

\begin{minted}[breaklines,fontsize=\scriptsize]{json}
    {
        "description":"Proof of Attendance to events hosted by NEAR Gang Couture.",
        "collection":"Haute Gang - Collaborations",
        "collection_id":"haute-gang-collaborations-by-neargangcouturenear",
        "creator_id":"neargangcouture.near",
        "attributes":[
            {"trait_type":"Rarity","value":"No Star"},
            {"trait_type":"Type","value":"Mask"}
        ],
        "blurhash":"UqFtJxPWpdyDGJ${t2V[?[ICMyenPCxVobae",
        "mime_type":"image/jpeg"
    }
\end{minted}

\begin{remark}
    Обычно smart-контракты DApps принято выкладывать в открытый доступ, чтобы любой пользователь мог их посмотреть и полностью доверять сервису.
\end{remark}


\paragraph{Mintbase}

Mintbase является менее популярным маркетплейсом, однако он предоставляем гораздо больше категорий NFT, но все ключевые функции такие же. В качестве
новых категорий выступают: 3d изображение, gif, профессиональные фотографии, аудиодорожки, произведения художников.

\subsection{Генеративно-состязательные сети}
