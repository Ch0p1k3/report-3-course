\section{Обзор существующих работ и решений}
\label{section.3}
\subsection{Маркетплейсы}
На данный момент существует большое количество NFT маркетплейсов: opensea\cite{opensea}, rarible\cite{rarible}, solanart\cite{solanart}. Если брать маркетплейсы только на базе NEAR Protocol, тогда существуют такие примеры как: Paras\cite{paras}, Mintbase\cite{mintbase}, остановимся на них поподробнее.

\paragraph{Paras}

Paras является наиболее популярным и представляет следующий набор функций:
посмотреть какие NFT выставлены на продажу, посмотреть купленные NFT, купить NFT, продать NFT, создать новую коллекцию NFT.
На площадке представлены следующие виды NFT: пиксель-арты, иллюстрации, абстрактные картины, картины разных персонажей, фотографии. Все объекты можно отсортировать по убыванию или возрастанию цены.

Smart-контракты Paras лежат в открытом доступе\cite{parasnftcontract, parasmarketplacecontract}.

\begin{remark}
    Обычно smart-контракты DApps принято выкладывать в открытый доступ, чтобы любой пользователь мог их посмотреть и полностью доверять сервису.
\end{remark}

// Тут что-то про smart-контракты

\begin{listing}
\begin{minted}[breaklines,fontsize=\scriptsize]{js}
{
    token_id: "304990:24",
    owner_id: "maxzeinly.near",
    metadata: {
        title: "Proof of Attendance No.1 #24",
        description: null,
        media: "bafybeib3c3r7vjbmyetawahj4kprei6satcrq23k2qjlx2gnmxmv5c6lza",
        media_hash: null,
        copies: 1111,
        issued_at: "1652813800358071368",
        expires_at: null,
        starts_at: null,
        updated_at: null,
        extra: null,
        reference: "bafkreiai54itp2hf267leg6754xmlst6j5m3yp3sin6n5bgva2q44wwtem",
        reference_hash: null
    },
    approved_account_ids: {}
}
\end{minted}
\caption{Структура NFT}
\end{listing}

Paras, как и большинство маркетплейсов хранит медиа-файл и метаданные NFT на IPFS\cite{ipfs}. IPFS предоставляется сервисом fleek\cite{fleek}. В качестве ссылки на медиа-файл и метаданные они хранят CID, а не полный URL, это связанно с тем, что минт NFT таким образом будет гораздо дешевле, ведь хранение в NEAR, довольно дорогое\cite{nearstoragestackinghowmuch}. Но есть и минус этой экономии, на NEAR Wallet, скорее всего это изображение не будет отображаться, так как NEAR Wallet, при не указании протокола соединения, будет подставлять CID не в URL IPFS от fleek\hyperref[lst.pastecidnearwallet]{(Листинг 2)}.

\begin{definition}
    IPFS(InterPlanetary File System) - децентрализованная система хранения файлов. При добавлении файла в IPFS, он делится на маленькие куски, криптографически хэшируется и отдается уникальный фингерпринтр, который называется CID(Content identifier) \cite{ipfs}
\end{definition}

\begin{listing}[H]
\begin{minted}[breaklines,fontsize=\scriptsize]{js}
function buildMediaUrl(media, base_uri) {
    if (!media || media.includes('://') || media.startsWith('data:image')) {
        return media;
    }
    if (base_uri) {
        return `${base_uri}/${media}`;
    }
    return `https://cloudflare-ipfs.com/ipfs/${media}`;
}
\end{minted}
\caption{Подстановка CID в URL у NEAR Wallet\cite{pastecidnearwallet}}
\label{lst.pastecidnearwallet}
\end{listing}

Давайте рассмотрим структуру метаданных NFT\hyperref[lst.parasnftmetadatastruct]{(Листинг 3)}. Paras, хоть и поддерживает по стандарту NEP-171 поле <<description>>, но хранит описание в метаданных NFT токена, это аналогично тем же причинам, что и хранение CID, а не полного url, в полях на медиа-файл и метаданные. Также они хранят название и идентификатор коллекции, создателя NFT, атрибуты и тип файла. Во многом мы будет подражать этой структуре в наших метаданных.

\begin{listing}[H]
\begin{minted}[breaklines,fontsize=\scriptsize]{json}
{
    "description":"Proof of Attendance to events hosted by NEAR Gang Couture.",
    "collection":"Haute Gang - Collaborations",
    "collection_id":"haute-gang-collaborations-by-neargangcouturenear",
    "creator_id":"neargangcouture.near",
    "attributes":[
        {"trait_type":"Rarity","value":"No Star"},
        {"trait_type":"Type","value":"Mask"}
    ],
    "blurhash":"UqFtJxPWpdyDGJ${t2V[?[ICMyenPCxVobae",
    "mime_type":"image/jpeg"
}
\end{minted}
\caption{Структура метаданных NFT в Paras}
\label{lst.parasnftmetadatastruct}
\end{listing}


\paragraph{Mintbase}

Mintbase является менее популярным маркетплейсом, однако он предоставляем гораздо больше категорий NFT, но все ключевые функции такие же. В качестве новых категорий выступают: 3D изображение, gif, профессиональные фотографии, аудиодорожки, произведения художников.

Smart-контракты Mintbase на половину открыты(некоторые в открытом доступе, некоторые нет)\cite{mintbasecontracts}.

// TODO контракты

\subsection{Генеративно-состязательные сети}
