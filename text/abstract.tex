\anonsection{Аннотация}

Первый глобальный старт блокчейна был в 2009 году с криптовалютой, которая называется Bitcoin - распределенная децентрализованная платежная система, разработанная Сатоси Накамото\footnote{Сатоси Накамото на самом деле псевдоним человека или группы людей. \url{https://en.wikipedia.org/wiki/Satoshi_Nakamoto}}. С тех самых пор многие приложения начали формировать так называемый web3. Если изначальная идея блокчейна была возможностью передачи электронной валюты по средствам peer-to-peer без каких-либо посредников, например, как банки, то на сегодняшний день блокчейн - это распределенный реестр, который представляет как аспекты вычислений, так и хранения данных, которые согласуются на основе консенсуса. Представленные возможности реализуются с помощью, так называемых, smart-контрактов - некоторые куски кода, которые выполняются непосредственно в блокчейне.

Вместе с концептами блокчейна и smart-контрактов нам пришло понятие NFT(non-fungible token). NFT - это уникальный криптографический токен, который не может быть замещен другим таким же, данный токен представляет некоторые цифровой объект - файл, текст или некоторый медиа объект. На идее NFT существует множество площадок, которые позволяют обмениваться данными токенами. Цель нашего проекта - реализовать свою такую площадку, ознакомиться с основными теоретическими понятиями связанных с этим и описать их в данном плане.