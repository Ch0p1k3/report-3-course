%%% Язык текста %%%
\selectlanguage{russian}

%%% Кодировки и шрифты %%%
%\usefont{T2A}{ftm}{m}{n}
\renewcommand{\rmdefault}{ftm} % Включаем Times New Roman Tempora-TLF

%%% Макет страницы %%%
\geometry{a4paper,top=15mm,left=25mm,right=15mm,bottom=15mm}
\setstretch{1.05}

%%% Содержание %%%
\renewcommand{\cfttoctitlefont}{\hfil \large\bfseries}

\setlength{\cftparskip}{0mm}
\setlength{\cftbeforesecskip}{0mm}
\setlength{\cftaftertoctitleskip}{\baselineskip}
\cftsetpnumwidth{4mm}

\renewcommand{\cftsecfont}{}
\renewcommand{\cftsecpagefont}{\normalsize}
\renewcommand{\cftsecleader}{\cftdotfill{\cftdotsep}}

\setlength{\cftsecindent}{0mm}
\setlength{\cftsecnumwidth}{6mm}

\setlength{\cftsubsecindent}{4mm}
\setlength{\cftsubsecnumwidth}{10mm}

\setlength{\cftsubsubsecindent}{12mm}
\setlength{\cftsubsubsecnumwidth}{12mm}

%%% Определения %%%
\newtheorem*{definition}{Определение}

%%% Замечание %%%
\newtheorem*{remark}{Замечание}

%%% Выравнивание и переносы %%%
\sloppy				% Избавляемся от переполнений
\clubpenalty=10000		% Запрещаем разрыв страницы после первой строки абзаца
\widowpenalty=10000		% Запрещаем разрыв страницы после последней строки абзаца
\interfootnotelinepenalty=10000 % Запрет разрывов сносок

%%% Настройки полей %%%

% Титульная страница
\fancypagestyle{empty}{%
\fancyhf{} % clear all header and footer fields
\renewcommand{\headrulewidth}{0pt}
\renewcommand{\footrulewidth}{0pt}
% \setlength{\footskip}{9mm}
\setlength{\headheight}{12.7mm}
}

% Основной текст
%\fancypagestyle{plain}{%
%\fancyhf{} % clear all header and footer fields
%\lhead{\fontsize{9}{6}\selectfont Е.А. Орешонок}
%\rhead{\fontsize{9}{6}\selectfont <<Изучение возможности применения понятия кривизны графа к данным мозговой активности>>}
%\fancyfoot[C]{\thepage}
%\renewcommand{\headrulewidth}{0pt}
%\renewcommand{\footrulewidth}{0pt}
%\setlength{\footskip}{11mm}
%\setlength{\headheight}{4mm}
%}

\fancypagestyle{jopa}{%
\fancyhf{} % clear all header and footer fields
\rhead{\fontsize{9}{6}\selectfont\color{gray} «Реализация NFT маркетплейса на базе Discord API»}
\fancyfoot[C]{\thepage}
\renewcommand{\headrulewidth}{0pt}
\renewcommand{\footrulewidth}{0pt}
\setlength{\footskip}{11mm}
\setlength{\headheight}{4mm}
}

%\pagestyle{plain}

%%% Оформление текста

\setlength{\parskip}{0mm}
\setlength{\parindent}{1cm}
\raggedbottom{}
\onehalfspacing

\renewcommand\floatpagefraction{.9}
\renewcommand\dblfloatpagefraction{.9} % for two column documents
\renewcommand\topfraction{.9}
\renewcommand\dbltopfraction{.9}       % for two column documents
\renewcommand\bottomfraction{.9}
\renewcommand\textfraction{.1}
\setcounter{totalnumber}{50}
\setcounter{topnumber}{50}
\setcounter{bottomnumber}{50}

%%% Оформление заголовков
%\newcommand{\sectionbreak}{\clearpage}

\newcommand{\anonsection}[1]{ \section*{#1} \addcontentsline{toc}{section}{
% \numberline {}
#1}}
\newcommand{\anonsubsection}[1]{ \subsection*{#1} \addcontentsline{toc}{subsection}{
% \numberline {}
#1}}
\newcommand{\anonsubsubsection}[1]{ \subsubsection*{#1} \addcontentsline{toc}{subsubsection}{
% \numberline {}
#1}}

\titleformat{\section}{\large\bfseries}{\thesection}{\wordsep}{}
% \titlespacing*{\section}{\parindent}{\baselineskip}{\baselineskip}

\titleformat{name=\section,numberless}{\large\bfseries}{}{0mm}{}
% \titlespacing*{name=\section,numberless}{1.25cm}{\baselineskip}{3mm}

\titleformat{name=\subsection}{\normalsize\bfseries}{\thesubsection}{\wordsep}{}
% \titlespacing*{\subsection}{\parindent}{\baselineskip}{\baselineskip}

\titleformat{name=\subsection,numberless}{\normalsize\bfseries}{}{0mm}{}
% \titlespacing*{name=\subsection,numberless}{0mm}{\baselineskip}{\baselineskip}

\titleformat{name=\subsubsection}{\normalsize\bfseries}{\thesubsubsection}{\wordsep}{}
% \titlespacing*{\subsubsection}{\parindent}{\baselineskip}{\baselineskip}

\titleformat{name=\subsubsection,numberless}{\normalsize\bfseries}{}{0mm}{}
% \titlespacing*{name=\subsubsection,numberless}{0mm}{\baselineskip}{\baselineskip}

\counterwithout{paragraph}{subsubsection}
\counterwithin{paragraph}{subsection}
\renewcommand{\theparagraph}{\thesubsection.\arabic{paragraph}}
\setcounter{secnumdepth}{4}

\titleformat{name=\paragraph}[runin]{\normalsize\bfseries}{\theparagraph}{\wordsep}{}
\titlespacing*{\paragraph}{\parindent}{\baselineskip}{\wordsep}

%%% Оформление списков
\setlist[1]{itemindent=1.85cm,leftmargin=0mm,itemsep=0mm,topsep=0mm,parsep=0mm}
% \setlist[itemize,1]{label=---}
\setlist[enumerate,1]{label=\arabic*.}

\setlist[2]{itemindent=2.5cm,leftmargin=0mm,itemsep=0mm,topsep=0mm,parsep=0mm}

% Cтиль для списков, на которые есть ссылки в тексте
\AddEnumerateCounter{\asbuk}{\@asbuk}{\cyrm}
\newlist{reflist}{enumerate}{1}
\setlist*[reflist,1]{label=\asbuk*)}
\setlist*[reflist,2]{label=\arabic*)}

%%% Оформление сносок

\deffootnote[1.65cm]{0mm}{1.25cm}{\textsuperscript{\thefootnotemark) }}
\renewcommand{\footnotesize}{\normalsize\selectfont}
\setlength{\footnotesep}{\parsep}

%%% Оформление ссылок
\urlstyle{same}

%%% Размеры текста формул %%%

\DeclareMathSizes{12}{12}{6}{4}

%%% Расстояние между формулами

\AtBeginDocument{%
  \setlength\abovedisplayskip{\baselineskip}%
  \setlength\belowdisplayskip{\baselineskip}%
  \setlength\abovedisplayshortskip{\baselineskip}%
  \setlength\belowdisplayshortskip{\baselineskip}%
}

%%% Расстояние между плавающими элементами

\setlength{\floatsep}{1\baselineskip plus 0mm minus 0mm}     % between top floats
\setlength{\textfloatsep}{1\baselineskip plus 0mm minus 0mm} % between top/bottom floats and text
\setlength{\intextsep}{1\baselineskip plus 0mm minus 0mm}    % between text and float
\setlength{\dbltextfloatsep}{1.5\baselineskip plus 0mm minus 0mm}
\setlength{\dblfloatsep}{1.5\baselineskip plus 0mm minus 0mm}

%% Нумерация плавающих элементов

\counterwithin{figure}{section}
\counterwithin{table}{section}
\counterwithin{listing}{section}
% \counterwithin{equation}{section}

\makeatletter
\AtBeginDocument{%
\renewcommand{\thetable}{\thesection.\arabic{table}}
\renewcommand{\thelstlisting}{\thesection.\arabic{lstlisting}}
\renewcommand{\thefigure}{\thesection.\arabic{figure}}
\let\c@lstlisting\c@figure}
\makeatother

%% Подписи плавающих элементов

\fboxsep=1mm
\fboxrule=0.1mm

\captionsetup[figure]{
  labelsep=period,
  justification=centering,
  singlelinecheck=false,
  position=bottom,
  parskip=\parskip,
  belowskip=-0.6\baselineskip,
  skip=1.4\baselineskip,
  font={it},
  labelfont={it},
  textfont={it}
}

\captionsetup[table]{
  position=top,
  labelsep=endash,
  justification=raggedright,
  singlelinecheck=false,
  position=top,
  belowskip=0.4\baselineskip,
  skip=0mm}

\captionsetup[longtable]{
  labelsep=endash,
  justification=raggedright,
  singlelinecheck=false,
  position=top,
  belowskip=0.4\baselineskip,
  skip=0mm}

\captionsetup[lstlisting]{
  labelsep=period
}

\lstset{
basicstyle=\scriptsize\ttfamily,
numberstyle=\scriptsize\ttfamily,
keywordstyle=\bfseries,
commentstyle=\itshape,
numbers=left,
stepnumber=1,
frame=single,
resetmargins=true,
xleftmargin=7mm,
xrightmargin=2mm,
captionpos=b,
keepspaces=true,
breaklines=true,
aboveskip=1.6\baselineskip,
belowskip=1.4\baselineskip,
abovecaptionskip=1.2\baselineskip}

\renewcommand{\arraystretch}{1.5}

%%% Настройка размеров вертикальных отступов

\renewcommand{\smallskip}{\vspace{0.3\baselineskip}}
\renewcommand{\bigskip}{\vspace{0.8\baselineskip}}

%%% Библиография %%%
% \makeatletter
% \bibliographystyle{sys/ugost2003} % Оформляем библиографию в соответствии с ГОСТ 7.1 2003

\let\oldthebibliography=\thebibliography
\let\endoldthebibliography=\endthebibliography
\renewenvironment{thebibliography}[1]{
  \begin{oldthebibliography}{#1}
    \setlength{\parskip}{0mm}
    \setlength{\itemsep}{0mm}
}
{
\end{oldthebibliography}
}


% Листинг

\renewcommand{\listingscaption}{Листинг}
